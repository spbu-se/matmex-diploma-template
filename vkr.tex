% !TEX TS-program = xelatex
% !BIB program = bibtex
% !TeX spellcheck = ru_RU

% About magic macroses see also
% https://tex.stackexchange.com/questions/78101/

% По умолчанию используется шрифт 14 размера. Если нужен 12-й шрифт, уберите опцию [14pt]
\documentclass[14pt, russian]{matmex-diploma-custom}

% !TeX spellcheck = ru_RU
% !TEX root = vkr.tex
% Опциональные добавления используемых пакетов. Вполне может быть, что они вам не понадобятся, но в шаблоне приведены примеры их использования.
\usepackage{tikz} % Мощный пакет для создание рисунков, однако может очень сильно замедлять компиляцию
\usetikzlibrary{decorations.pathreplacing,calc,shapes,positioning,tikzmark}

% Библиотека для TikZ, которая генерирует отдельные файлы для каждого рисунка
% Позволяет ускорить компиляцию, однако имеет свои ограничения
% Например, ломает пример выделения кода в листинге из шаблона
% \usetikzlibrary{external}
% \tikzexternalize[prefix=figures/]

\newcounter{tmkcount}

\tikzset{
    use tikzmark/.style={
            remember picture,
            overlay,
            execute at end picture={
                    \stepcounter{tmkcount}
                },
        },
    tikzmark suffix={-\thetmkcount}
}

\usepackage{booktabs} % Пакет для верстки "более книжных" таблиц, вполне годится для оформления результатов
% В шаблоне есть команда \multirowcell, которой нужен этот пакет.
\usepackage{multirow}
\usepackage{siunitx} % для таблиц с единицами измерений

% Для названий стоит использовать \textsc{}
\newcommand{\OCaml}{\textsc{OCaml}}
\newcommand{\miniKanren}{\textsc{miniKanren}}
\newcommand{\BibTeX}{\textsc{BibTeX}}
\newcommand{\vsharp}{\textsc{V$\sharp$}}
\newcommand{\fsharp}{\textsc{F$\sharp$}}
\newcommand{\csharp}{\textsc{C\#}}
\newcommand{\GitHub}{\textsc{GitHub}}
\newcommand{\SMT}{\textsc{SMT}}

\definecolor{eclipseGreen}{RGB}{63,127,95}
% \lstdefinelanguage{ocaml}{
% keywords={@type, function, fun, let, in, match, with, when, class, type,
% nonrec, object, method, of, rec, repeat, until, while, not, do, done, as, val, inherit, and,
% new, module, sig, deriving, datatype, struct, if, then, else, open, private, virtual, include, success, failure,
% lazy, assert, true, false, end},
% sensitive=true,
% commentstyle=\small\itshape\ttfamily,
% keywordstyle=\ttfamily\bfseries, %\underbar,
% identifierstyle=\ttfamily,
% basewidth={0.5em,0.5em},
% columns=fixed,
% fontadjust=true,
% literate={->}{{$\to$}}3 {===}{{$\equiv$}}1 {=/=}{{$\not\equiv$}}1 {|>}{{$\triangleright$}}3 {\\/}{{$\vee$}}2 {/\\}{{$\wedge$}}2 {>=}{{$\ge$}}1 {<=}{{$\le$}} 1,
% morecomment=[s]{(*}{*)}
% }

\makeatletter
\@ifclassloaded{beamer}{
    %%% Обязательные пакеты
    %% Beamer
    \usepackage{beamerthemesplit}
    \usetheme{SPbGU}
    \beamertemplatenavigationsymbolsempty
    \usepackage{appendixnumberbeamer}

    %% Локализация
    \usepackage{fontspec}
    \setmainfont{CMU Serif}
    \setsansfont{CMU Sans Serif}
    \setmonofont{CMU Typewriter Text}
    %\setmonofont{Fira Code}[Contextuals=Alternate,Scale=0.9]
    %\setmonofont{Inconsolata}
    \usepackage{polyglossia}
    \setmainlanguage{russian}
    \setotherlanguage{english}

    %% Графика
    \usepackage{pdfpages} % Позволяет вставлять многостраничные pdf документы в текст

    % Математические окружения с русским названием
    \newtheorem{rutheorem}{Теорема}
    \newtheorem{ruproof}{Доказательство}
    \newtheorem{rudefinition}{Определение}
    \newtheorem{rulemma}{Лемма}
    \usepackage{fancyvrb}
}
{}
\makeatother

\usepackage[autostyle]{csquotes} % Правильные кавычки в зависимости от языка
\usepackage{totcount}
\usepackage{setspace}
\usepackage{amsmath, amsfonts, amssymb, amsthm, mathtools} % "Адекватная" работа с математикой в LaTeX



\begin{document}
% TODO: Formatting
%% Если что-то забыли, при компиляции будут ошибки Undefined control sequence \my@title@<что забыли>@ru
%% Если англоязычная титульная страница не нужна, то ее можно просто удалить.
\filltitle{ru}{
    %% Актуально только для курсовых/практик. ВКР защищаются не на кафедре а в ГЭК по направлению, 
    %%   и к моменту защиты вы будете уже не в группе.
    chair              = {Кафедра, на которой работает научник},
    group              = {ХХБ.ХХ-мм},
    %% Макрос filltitle ненавидит пустые строки, поэтому обязателен хотя бы символ комментария на строке
    %% Актуально всем.
    title              = {Шаблон отчёта по учебной практике},
    % 
    %% Здесь указывается тип работы. Возможные значения:
    %%   coursework - отчёт по курсовой работе;
    %%   practice - отчёт по учебной практике;
    %%   prediploma - отчёт по преддипломной практике;
    %%   master - ВКР магистра;
    %%   bachelor - ВКР бакалавра.
    type               = {practice},
    author             = {ФАМИЛИЯ Имя Отчество},
    % 
    %% Актуально только для ВКР. Указывается код и название направления подготовки. Типичные примеры:
    %%   02.03.03 <<Математическое обеспечение и администрирование информационных систем>>
    %%   02.04.03 <<Математическое обеспечение и администрирование информационных систем>>
    %%   09.03.04 <<Программная инженерия>>
    %%   09.04.04 <<Программная инженерия>>
    %% Те, что с 03 в середине --- бакалавриат, с 04 --- магистратура.
    specialty          = {02.03.03 <<Математическое обеспечение и администрирование информационных систем>>},
    % 
    %% Актуально только для ВКР. Указывается шифр и название образовательной программы. Типичные примеры:
    %%   СВ.5006.2017 <<Математическое обеспечение и администрирование информационных систем>>
    %%   СВ.5162.2020 <<Технологии программирования>>
    %%   СВ.5080.2017 <<Программная инженерия>>
    %%   ВМ.5665.2019 <<Математическое обеспечение и администрирование информационных систем>>
    %%   ВМ.5666.2019 <<Программная инженерия>>
    %% Шифр и название программы можно посмотреть в учебном плане, по которому вы учитесь. 
    %% СВ.* --- бакалавриат, ВМ.* --- магистратура. В конце --- год поступления (не обязательно ваш, если вы были в академе/вылетали).
    programme          = {СВ.5006.2017 <<Математическое обеспечение и администрирование информационных систем>>},
    % 
    %% Актуально только для ВКР, только для матобеса и только 2017-2018 годов поступления. Указывается профиль подготовки, на котором вы учитесь.
    %% Названия профилей можно найти в учебном плане в списке дисциплин по выбору. На каком именно вы, вам должны были сказать после второго курса (можно уточнить в студотделе).
    %% Вот возможные вариканты:
    %%   Математические основы информатики
    %%   Информационные системы и базы данных
    %%   Параллельное программирование
    %%   Системное программирование
    %%   Технология программирования
    %%   Администрирование информационных систем
    %%   Реинжиниринг программного обеспечения
    % profile            = {Системное программирование},
    % 
    %% Актуально всем.
    %supervisorPosition = {проф. каф. СП, д.ф.-м.н., проф.}, % Терехов А.Н.
    supervisorPosition = {к.ф.-м.н., доцент кафедры информатики,}, % Григорьев С.В.   
    supervisor         = {Н.Н. Научник},  
    % 
    %% Актуально только для практик и курсовых. Если консультанта нет, закомментировать или удалить вовсе.
    consultantPosition = {должность ООО <<Место работы>> степень},
    consultant         = {К.К. Консультант},
    %
    %% Актуально только для ВКР.
    reviewerPosition   = {должность ООО <<Место работы>> степень},
    reviewer           = {Р.Р. Рецензент},
}

% \filltitle{en}{
%     chair              = {Advisor's chair},
%     group              = {ХХB.ХХ-mm},
%     title              = {Template for SPbU qualification works},
%     type               = {practice},
%     author             = {FirstName Surname},
%     % 
%     %% Possible choices:
%     %%   02.03.03 <<Software and Administration of Information Systems>>
%     %%   02.04.03 <<Software and Administration of Information Systems>>
%     %%   09.03.04 <<Software Engineering>>
%     %%   09.04.04 <<Software Engineering>>
%     %% Те, что с 03 в середине --- бакалавриат, с 04 --- магистратура.
%     specialty          = {02.03.03 ``Software and Administration of Information Systems''},
%     % 
%     %% Possible choices:
%     %%   СВ.5006.2017 <<Software and Administration of Information Systems>>
%     %%   СВ.5162.2020 <<Programming Technologies>>
%     %%   СВ.5080.2017 <<Software Engineering>>
%     %%   ВМ.5665.2019 <<Software and Administration of Information Systems>>
%     %%   ВМ.5666.2019 <<Software Engineering>>
%     programme          = {СВ.5006.2017 ``Software and Administration of Information Systems''},
%     % 
%     %% Possible choices:
%     %%   Mathematical Foundations of Informatics
%     %%   Information Systems and Databases
%     %%   Parallel Programming
%     %%   System Programming
%     %%   Programming Technology
%     %%   Information Systems Administration
%     %%   Software Reengineering
%     % profile            = {Software Engineering},
%     % 
%     %% Note that common title translations are:
%     %%   кандидат наук --- C.Sc. (NOT Ph.D.)
%     %%   доктор ... наук --- Sc.D.
%     %%   доцент --- docent (NOT assistant/associate prof.)
%     %%   профессор --- prof.
%     supervisorPosition = {Sc.D, prof.},
%     supervisor         = {S.S. Supervisor},
%     % 
%     consultantPosition = {position at ``Company'', degree if present},
%     consultant         = {C.C. Consultant},
%     %
%     reviewerPosition   = {position at ``Company'', degree if present},
%     reviewer           = {R.R. Reviewer},
% }
\maketitle
\setcounter{tocdepth}{2}
\tableofcontents

% \begin{abstract}
%   В курсаче не нужен
% \end{abstract}

\pagebreak
\begin{center}
  {\Huge
    Текст ВКР или учебной практики пишется не ради зачета, а чтобы люди его прочитали, поняли как круто Вы все сделали, и могли продолжить с того места, где Вы остановились.}

  \vspace{2em}
  Повторять эту страницу в тексте вашей работы нельзя.
\end{center}
\pagebreak

% !TeX spellcheck = ru_RU
% !TEX root = vkr.tex

\section*{Введение}
\thispagestyle{withCompileDate}

Формат из 4х частей рекомендуется в курсе Д.~Кознова~\cite{koznov} по написанию текстов.

\begin{enumerate}
    \item Известная информация (background/обзор).
    \item Неизвестная информация (пробел в знаниях, \enquote{Gap}).
    \item Гипотезы, вопросы, цели~--- \enquote{что болит}, что будет решать Ваша работа.
    \item Подход, план решения задачи, предлагаемое решение.
\end{enumerate}

Последний абзац должен читаться и быть понятен в отрыве от других трёх.
Никакие абзацы нумеровать нельзя.

Части (абзацы) должны занять максимум две страницы, идеально уложиться в одну.

С.-П. Джонс~\cite{SPJGreatPaper} предлагает несколько другой формат написания введения.
Вполне возможно, что если Ваша работа про языки программирования, то его формат будет удачнее.

Введение и заключение читают чаще всего, поэтому они должны быть \enquote{вылизаны} до блеска.

\blfootnote{
    Иногда рецензенту полезно знать какого числа компилировался текст, чтобы оценить актуальность версии текста.
    В этом случае полезно вставлять в текст дату сборки.
    Для совсем официальных релизов документа это не вполне канон.\\
    Также здесь имеет смысл указать, если работа сделана на деньги, например, Российского Фонда Фундаментальных Исследований (РФФИ) по гранту номер такой-то, и т.п.}

% !TeX spellcheck = ru_RU
% !TEX root = vkr.tex

\section{Постановка задачи}
\label{sec:task}

Дословно \enquote{Целью работы является... Для её выполнения были постав\-лены следующие задачи:}
\begin{enumerate}
    \item реализовать это (раздел~\ref{subsec:task1});
    \item спроектировать то-то (раздел~\ref{subsec:task2}) наилучшим образом;
    \item протестировать на том-то (раздел~\ref{subsec:task3}) и обогнать тех-то;
    \item \sout{изучить язык \OCaml{}} писать тут не надо, так как тут должны быть задачи, выполнение которых можно проверить/оценить прочитав текст или выслушав доклад;
          (т.е. Ваши достижения должны быть опровержимы)
          \begin{itemize}
              \item это может вызвать сомнения по поводу обзора~--- \emph{выполнить обзор} писать можно и нужно, но защищаемым результатом будут не ваши знания, а текст обзора (то есть он должен иметь ценность сам по себе);
          \end{itemize}
    \item обязательна задача на валидацию результата, будь то эксперимент, апробация, внедрение~--- то есть доказательство того, что Вы сделали что-то, нужное пользователю.
          Не путайте с валидацией~--- доказательством того, что Вы сделали то, что хотели Вы (например, тесты~--- валидация результата, хорошо, но недостаточно).
\end{enumerate}

% !TeX spellcheck = ru_RU
% !TEX root = vkr.tex

\section{Обзор (обязателен к новому году)}
\label{sec:relatedworks}
\emph{Обзор должен быть.} Здесь нужно писать, что индустрия и наука уже сделали по вашей теме. Он нужен, чтобы Вы случайно не изобрели какой-нибудь велосипед.

По-английски называется related works или previous works.

Если Ваша работа является развитием предыдущей и плохо понима\-ема без неё, то обзор должен идти в начале. Если же Вы решаете некоторую задачу новым интересным способом, то если поставить обзор в начале, то читатель может устать, пока доберется до вашего решения. В этом случае уместней поставить обзор после описания Вашего подхода к проблеме.

В обзоре необходимо ссылаться на работы других людей. В данном шаблоне задумано, что литература будет указываться в файле \verb=vkr.bib=. В нём указываются пункты литературы в формате \BibTeX{}, а затем на них можно ссылаться с помощью \verb=\cite{...}=. Та литература, на которую Вы сошлетесь, попадет в список литературы в конце документа. Если не сошлетесь~---  не попадёт. Спецификацию в формате \BibTeX{} почти никогда (для второго курса~--- никогда), не нужно придумывать руками. Правильно: находить в интернете описание цитируемой статьи\footnote{Например, \url{https://dl.acm.org/doi/10.1145/3408995} (дата доступа:   \DTMdate{2022-12-17}).},
копировать цитату с помощью кнопки \foreignquote{english}{Export Citation} и вставлять в \BibTeX{} файл. Если так не делать, но оформление литературы будет обрастать багами.
Например, \BibTeX{} по особенному обрабатывает точ\-ки, запятые и \verb=and= в списке авторов, что позволяет ему самому понимать, сколько авторов у статьи, и что там фамилия, что~--- имя, а что~--- отчество.

В обзоре и в остальном тексте вы наверняка будете использовать названия продуктов или языков программирования. Для них рекоменду\-ется (в файле \verb=preamble2.tex=) за\-дать специальные команды, чтобы писать сложные названия правильно и одинаково по всему доку\-менту. Написать с ошибкой  название любимого языка программирова\-ния науч\-ного руко\-водителя~--- идеальный вариант его выбесить.

% !TeX spellcheck = ru_RU
% !TEX root = vkr.tex

\section{Background (опционально)}
Здесь пишется некоторая дополнительная информация о том, зачем делается то, что делается.

Например, в работе придумывается какой-то новый метод решения формул в \SMT{} в теориях с числами. Без каких-то дополнительных пояснений будет казаться, что работа состоит из жестокого \enquote{матана} и совсем не по теме кафедры системного программирования.
Поэтому, в данном разделе стоит рассказать, что все эти методы примеряются для верификации в проекте \vsharp{}, и поэтому непосредственно связаны с тематикой кафедры.

% !TeX spellcheck = ru_RU
% !TEX root = vkr.tex

\section{Метод}
Реализация в широком смысле: что таки было сделано. Скорее всего самый большой раздел.

\emph{Крайне желательно} к Новому году иметь что-то, что сюда можно написать.

Для понимания того как курсовая записка (отзыв по учебной практи\-ке/ВКР) должна писаться, можно посмотреть видео ниже. Они про научные доклады и написание научных статей, учебные практики и ВКР отличаются тем, что тут есть требования отдельных глав (слайдов) со списком задач и списком результатов. Но даже если вы забьёте на требования специфичные для ВКР, и соблюдете все рекомендации ниже, получившиеся скорее всего будет лучше чем первая попытка 99\% ваших однокурсников.

\begin{enumerate}
\item \enquote{Как сделать великолепный научный доклад} от Саймона Пейтона Джонса~\cite{SPJGreatTalk} (на английском).
\item \enquote{Как написать великолепную научную статью} от Саймона Пейтона Джонса~\cite{SPJGreatPaper} (на английском).
\item \enquote{Как писать статьи так, чтобы люди их смогли прочитать} от Дэрэка Драйера~\cite{DreyerYoutube2020} (на английском). По предыдующей ссылке разбираются, что должно быть в статье, т.е. как она должна выгля\-деть на высоком уровне. Тут более низкоуровнево изложено как должны быть устроены параграфы и т.п.
\item Ещё можно посмотреть How to Design Talks~\cite{JhalaYoutube2020} от Ranjit Jhala, но я думаю, что первых трех ссылок всем хватит.
\end{enumerate}

\subsection{Первая задача}
\label{subsec:task1}

\subsection{Вторая задача}
\label{subsec:task2}

\subsection{Третья задача}
\label{subsec:task3}


\subsection{Некоторые типичные ошибки}
Здесь мы будем собирать основные ошибки, которые случаются при написании текстов.
В интернетах тоже можно найти коллекции типич\-ных косяков\footnote{\href{https://www.read.seas.harvard.edu/~kohler/latex.html}{https://www.read.seas.harvard.edu/\textasciitilde kohler/latex.html} (дата доступа:   \DTMdate{2022-12-16}).}.

Рекомендуется по-умол\-ча\-нию использовать красивые греческие бук\-вы $\sigma$  и $\phi$, а именно $\phi$ вместо $\varphi$. В данном шаблоне команды для этих букв переставлены местами по сравнению с ванильным \TeX'ом.

Также, если работа пишется на русском языке, необходимо, чтобы работа была написана на \textit{грамотном} русском языке даже, если автор, из ближнего зарубежья\footnote{
Теоретически, возможен вариант написания текстов на английском языке, но это необходимо обсудить в первую очередь с научным руководителем.}.
Это включает в себя:
\begin{itemize}
  \item разделы должны оформляться с помощью \verb=\section{...}=, а также \verb=\subsection= и т.~п.; не нужно пытаться нумеровать вручную;
  \item точки после окончания предложений должны присутствовать;
  \item пробелы после запятых  и точек, в конце слова перед скобкой;
  \item неразрывные пробелы, там, где нужны пробелы, но переносить на другую строку нельзя, например \verb=т.~е.= или \verb=А.~Н.~Терехов=;
  \item дефис, там где нужен дефис (обозначается с помощью одиночного \enquote{минуса} (англ. dash) на клавиатуре);
  \item двойной дефис, там где он нужен; а именно  при указании проме\-жутка в цифрах: в 1900--1910 г.~г., стр. 150--154;
  \item даты стоит писать везде одинаково; чтобы об этом не следить, можно пользоваться заклинанием \verb=\DTMdate{2022-12-16}=;
  \item тире (т.~е. \verb=---= --- тройной минус) на месте тире, а не что-то другое; в русском языке тире не может \enquote{съезжать} на новую строку, поэтому стоит использовать такой синтаксис: \verb=До~--- после=;
  \item правильные двойные кавычки должны набираться с помощью пакета \texttt{csquotes}: для основного языка в \texttt{polyglossia} стоит использовать команду \verb=\enquote{текст}=, для второго языка стоит использовать \verb=\foreignquote{язык}{текст}=;
  \item все перечисления должны оформляться с помощью \verb=\enumerate= или \verb=\itemize=; пункты перечислений должны либо начинаться с заглавной буквой и заканчиваться точкой, либо начинаться со строчной и закачиваться точкой с запятой; последний пункт пере\-числений всегда заканчивается точкой.
  \item Перед выкладкой финальной версии необходимо просмотреть лог \LaTeX'a, и обратить внимание на сообщения вида \emph{Overfull \textbackslash hbox (1.29312pt too wide) in paragraph}. Обычно, это означает, что текст выползает за поля, и надо подсказать, как правильно слова на слоги, чтобы перенос произошел автоматически. Это делается, например, так: \verb=соломо\-волокуша=.
\end{itemize}



\subsection{Листинги, картинка и прочий \enquote{не текст}}

Различный \enquote{не текст} имеет свойство отображаться не там, где он написан в текстовом в \LaTeX{}, поэтому у него должна быть самодостаточ\-ная понятная подпись \verb=\caption{...}=, уникальная метка \verb=\label{...}=, чтобы на неё можно было бы ссылаться в тексте с помощью \verb=\ref{...}=. Ниже вы сможете увидеть таблицу \ref{time_cmp_obj_func}, на которую мы сослались буквально только что.

\enquote{Не текста} может быть довольно много, чтобы не засорять корневую папку, хорошим решением будет складывать весь \enquote{не текст} в папку \texttt{figures}.
Заклинание \verb=\includegraphics{}= уже знает этот путь и будет искать там файлы без указания папки.
Команда \verb=\input{}= умеет ходит по путям, например \verb=\input{figures/my_awesome_table.tex}=.
Кроме того, листинги кода можно подтягивать из файла с помощью команды \verb=\lstinputlisting{file}=.
%% TODO: Проверить, что на Windows \input{folder/file} работает, если нет, использовать пакет import

%% Вставка кода с помощью listings
\begin{lstlisting}[caption={Название для листинга кода. Достаточно длинное, чтобы люди, которые смотрят картинку сразу после названия статьи (т.~е. все люди), смогли разобраться и понять к чему в статье листинги, картинки и прочий \enquote{не текст}.}, language=Caml, frame=single]
  let x = 5 in x+1
\end{lstlisting}
%% Вставка кода с помощью minted
% \begin{listing}
%   \caption{Название для листинга кода. Достаточно длинное, чтобы люди, которые смотрят картинку сразу после названия статьи (т.~е. все люди), смогли разобраться и понять к чему в статье листинги, картинки и прочий \enquote{не текст}.}
%   \begin{minted}[frame=single]{ocaml}
%     let x = 5 in x+1
%   \end{minted}
% \end{listing}



\subsubsection{Выделения куска листинга с помощью tikz}
Это обывает полезно в текста, а ещё чаще~--- в презентациях. Пример сделан на основе вопроса на \textsc{StackOveflow}\footnote{\url{https://tex.stackexchange.com/questions/284311} (дата доступа:   \DTMdate{2022-12-16}).}.

\begin{figure}
% TODO(Kakadu): Сделать \lstset глобально, чтобы не выписывать все опции листингов каждый раз
\begin{lstlisting}[escapechar=!, basicstyle=\ttfamily, language=c, keepspaces=true]
#include <stdio.h>
#include <math.h>

int main(void)
{
  double c = -1;
  double z = 0;

  (* Это комментарий на русском языке *)
  printf ("For c = %lf:\n", c);
  for (int i=0; i<10; i++ ) {
    printf ( !\tikzmark{a}!"z %d = %lf\n"!\tikzmark{b}!, i, z);
    z = pow(z, 2) + c;
  }
}
\end{lstlisting}

\begin{tikzpicture}[use tikzmark]
\draw[fill=gray,opacity=0.1]
  ([shift={(-3pt,2ex)}]pic cs:a)
    rectangle
  ([shift={(3pt,-0.65ex)}]pic cs:b);
\end{tikzpicture}
\caption{Пример листинга и \textsc{TIKZ} декорации к нему, оформленные в окружении \texttt{figure}. Обратите внимание, что рисунок отображается не там, где он в документе, а может \enquote{плавать}.}
\end{figure}

% !TeX spellcheck = ru_RU
% !TEX root = vkr.tex

\section{Эксперимент (желательно к Новому году)}
Как мы проверяем, что  всё удачно получилось.  К Новому году для промежуточного отчета желательно хотя бы описать как он будет прово\-диться и на чем.

\subsection{Условия эксперимента}
Железо (если актуально);  версии ОС, компиляторов и параметры командной строки; почему мы выбрали именно эти тесты; входные дан\-ные, на которых проверяем наш подход, и почему мы выбрали именно их.

\subsection{Исследовательские вопросы }
По-английски называется \emph{research questions}, в тексте можно ссылаться на них как RQ1, RQ2, и т.~д.
Необходимо сформулировать, чего мы хотели бы добиться работой (2 пункта будет хорошо):

\begin{itemize}
\item Хотим алгоритм, который лучше вот таких-то остальных.
\item Если в подходе можно включать/выключать составляющие, то насколько существенно каждая составляющая влияет на улучшения.
\item Если у нас строится приближение каких-то штук, то на сколько точными будут эти приближения.
\item и т.п.
\end{itemize}

Иногда в работах это называют гипотезами, которые потом проверяют. Далее в тексте можно ссылаться на research questions как \textsc{RQ}, это обще\-при\-нятое сокращение.

\subsection{Метрики}

Как мы сравниваем, что результаты двух подходов лучше или хуже:
\begin{itemize}
\item Производительность.
\item Строчки кода.
\item Как часто алгоритм \enquote{угадывает} правильную класси\-фикацию входа.
\end{itemize}

\noindent Иногда метрики вырожденные (да/нет), это не очень хорошо, но если в области исследований так принято, то ладно.

\subsection{Результаты}
Результаты понятно что такое. Тут всякие таблицы и графики, как в таблице \ref{time_cmp_obj_func}. Обратите внимание, как цифры выровнены по правому краю, названия по центру, а разделители $\times$ и $\pm$ друг под другом.

Скорее всего Ваши измерения будут удовлетворять нормальному распределению, в идеале это надо проверять с помощью критерия Кол\-могорова и т.п.
Если критерий этого не подтверждает, то у Вас что-то сильно не так с измерениями, надо проверять кэши процессора, отключать Интернет во время измерений, подкручивать среду исполне\-ния (англ. runtime), что\-бы сборка мусора не вмешивалась и т.п.
Если критерий удовлетворён, то необходимо либо указать мат. ожидание и доверительный/предсказы\-вающий интервал, либо написать, что все измерения проводились с погрешностью, например, в 5\%.
Замечание: если у вас получится улуч\-шение производительности в пределах погреш\-ности, то это обязательно вызовет вопросы.

В этом разделе надо также коснуться Research Questions.

\subsubsection{RQ1} Пояснения
\subsubsection{RQ2} Пояснения

\begin{table}
\def\arraystretch{1.1}  % Растяжение строк в таблицах
\setlength\tabcolsep{0.2em}
\centering
% \resizebox{\linewidth}{!}{%
    \caption{Производительность какого-то алгоритма при различных разрешениях картинок  (меньше~--- лучше), в мс.,  CI=0.95. За пример таблички кидаем чепчики в честь Я.~Кириленко}
    \begin{tabular}[C]{
    S[table-format=4.4,output-decimal-marker=\times]
    *4{S
          [table-figures-uncertainty=2, separate-uncertainty=true, table-align-uncertainty=true,
          table-figures-integer=3, table-figures-decimal=2, round-precision=2,
          table-number-alignment=center]
          }
    }
    \toprule
        \multicolumn{1}{r}{Resolution} & \multicolumn{1}{r}{\textsc{TENG}} & \multicolumn{1}{r}{\textsc{LAPM}} &
        \multicolumn{1}{r}{\textsc{VOLL4}} \\ \midrule
        1920.1080 & 406.23 \pm 0.94 & 134.06 \pm 0.35 & 207.45 \pm 0.42  \\ \midrule
        1024.768  & 145.0 \pm 0.47  & 39.68 \pm 0.1   &  52.79  \pm 0.1 \\ \midrule
        464.848   & 70.57 \pm 0.2   & 19.86 \pm 0.01     & 32.75  \pm 0.04 \\ \midrule
        640.480   & 51.10 \pm 0.2   & 14.70 \pm 0.1 & 24  \pm 0.04 \\ \midrule
        160.120   & 2.4 \pm 0.02    & 0.67 \pm 0.01      & 0.92  \pm 0.01 \\
        \bottomrule
    \end{tabular}%
%}
    \label{time_cmp_obj_func}
\end{table}

\clearpage
% !TeX spellcheck = ru_RU
% !TEX root = vkr.tex

\newcolumntype{C}{ >{\centering\arraybackslash} m{4cm} }
\newcommand\myvert[1]{\rotatebox[origin=c]{90}{#1}}
\newcommand\myvertcell[1]{\multirowcell{5}{\myvert{#1}}}
\newcommand\myvertcelll[1]{\multirowcell{4}{\myvert{#1}}}
\newcommand\myvertcellN[2]{\multirowcell{#1}{\myvert{#2}}}

\afterpage{%
    \clearpage% Flush earlier floats (otherwise order might not be correct)
    \thispagestyle{empty}% empty page style (?)
    \begin{landscape}% Landscape page
        \centering % Center table

        \begin{tabular}{|c|c|c|c|c|c|c|c|c|c|c|c|c|c|c|c|c|c|}\hline
            %& \multicolumn{17}{c|}{} \\ \hline
            \multirowcell{2}{Код модуля \\в составе \\ дисциплины,\\практики и т.п. }
            &\myvertcellN{2}{Трудоёмкость\quad}
            & \multicolumn{10}{c|}{\tiny{Контактная работа обучающихся с преподавателем}}
            & \multicolumn{5}{c|}{\tiny{Самостоятельная работа}}
            & \myvertcellN{2}{\tiny Объем активных и интерактивных\quad}
            \\ \cline{3-17}

            && \myvertcellN{2}{лекции\quad}
            &\myvertcellN{2}{семинары\quad}
            &\myvertcellN{2}{консультации\quad}
            &\myvertcellN{2}{\small практические  занятия\quad}
            &\myvertcellN{2}{\small лабораторные работы\quad}
            &\myvertcellN{2}{\small контрольные работы\quad}
            &\myvertcellN{2}{\small коллоквиумы\quad}
            &\myvertcellN{2}{\small текущий контроль\quad}
            &\myvertcellN{2}{\small промежуточная аттестация\quad}
            &\myvertcellN{2}{\small итоговая аттестация\quad}

            &\myvertcellN{2}{\tiny под руководством    преподавателя\quad}
            &\myvertcellN{2}{\tiny в присутствии     преподавателя\quad}
            &\myvertcellN{2}{\tiny с использованием    методических\quad}
            &\myvertcellN{2}{\small текущий контроль\quad}
            &\myvertcellN{2}{\makecell{\small промежуточная \\ аттестация}}
            &     \\
            && &&&&&&&&& &&&&&&\\
            && &&&&&&&&& &&&&&&\\
            && &&&&&&&&& &&&&&&\\
            &&&&&&&&&&& &&&&&&\\
            &&&&&&&&&&& &&&&&&\\
            &&&&&&&&&&& &&&&&&\\ \hline
            Семестр 3 & 2 &30  &&&&&&&&2   & &&&18 &&20 &10\\ \hline
            &   &2--42&&&&&&&&2--25& &&&1--1&&1--1&\\ \hline
            Итого     & 2 &30  &&&&&&&&2   & &&&18 &&20 &10\\ \hline
        \end{tabular}

        \captionof{table}{Если таблица очень большая, то можно её изобразить на отдельной портретной странице. Не забудьте подробное описание, чтобы содержимое таблицы можно было понять не читая весь текст.}
    \end{landscape}
    \clearpage% Flush page
}



\subsection{Обсуждение результатов}

Чуть более неформальное обсуждение, то, что сделано. Например, почему метод работает лучше остальных? Или, что делать со случаями, когда метод классифицирует вход некорректно.

% !TeX spellcheck = ru_RU
% !TEX root = vkr.tex

\section{Применение (того, что сделано на практике)}

Если применение в лоб не работает, потому что всё изложено чуть более сжато и теоретично, надо рассказать тонкости и правильный метод применения результатов. Если результаты применяются без до\-полнительных телодвижений, то про это можно не писать.

% !TeX spellcheck = ru_RU
% !TEX root = vkr.tex

\section{Угрозы нарушения корректности (опциональный)}

Если основная заслуга метода, это то, что он дает лучшие цифры, то стоит сказать, где мы могли облажаться, когда
\begin{enumerate}
  \item проводили численные замеры;
  \item выбирали тестовый набор (см. \emph{confirmation bias}).
\end{enumerate}

% !TeX spellcheck = ru_RU
% !TEX root = vkr.tex

\section{Реализация}
Очень важный раздел для будущих программных инженеров, т.е. почти для всех. Важно иметь всегда, в том числе для промежуточных отчетов по учебным практикам или ВКР.

В процессе работы можно сделать огромное количество косяков, неполный список которых ниже.

\begin{enumerate}
  \item Реализация должна быть. На публично доступную реализацию обязательная ссылка. Если код под \textsc{NDA}, то об этом,
  во-первых, должно быть сказано явно,
  во-вторых, на защиту должны выно\-ситься другие результаты (например, архитектура), чтобы комис\-сия имела возможность оценить хоть что-то,
  и, в третьих, должна быть справка от работодателя, что вы правда что-то сделали.
        \begin{itemize}
          \item  Рецензент обязан оценить код (о возможности должен побеспо\-коиться обучающийся).
        \end{itemize}
  \item Код реализации должен быть написан защищающимся целиком.
        \begin{itemize}
          \item  Если проект групповой, то нужно явно выделить какие части были модифицированы защищающимся. Например, в преды\-дущих разделах на картинке архитектуры нужно выделить цветом то, что вы модифицировали.
          \item Нельзя пускать в негрупповой проект коммиты от других людей, или людей не похожих на Вас. Например, в 2022 году защищающийся-парень делал коммиты от сценического псев\-донима, который намекает на женский \enquote{гендер}. (Нет, это не шутка.) На тот момент в российской культуре это выглядело странно.
          \item Возможна ситуация, что вы используете конкретный ник в интернете уже лет пять, и желаете писать ВКР под этим ником на \GitHub{}. В принципе, это допустимо (не только лишь я так считаю), но если Вы встретите преподавателя, который считает наоборот, то Вам придется грамотно отмазы\-ваться. В Вашу пользу могут сыграть те факты, что к нику на гитхабе у Вас приписаны настоящие имя и фамилия; что в репозитории у вас видна домашка за первый курс; и что Ваш преподаватель практики сможет подтвердить, что Вы уже несколько лет используете это ник; и т.п.
        \end{itemize}
  \item Если вы получаете диплом о присвоении звания программного инженера, код должен соответствовать.
        \begin{enumerate}
          \item Не стоит выкладывать код одним коммитом.
          \item Не стоит выкладывать код аккурат перед защитой.
          \item Лучше хоть какие-то тесты, чем совсем без них. В идеале нужно предъявлять процент покрытия кода тестами.
          \item Лучше  сделать \textsc{CI}, а также \textsc{CD}, если оно уместно в Вашем проекте.
          \item Не стоит демонстрировать на защите, что Вам даже не пришло в голову напустить на код линтеры и т.п.
        \end{enumerate}
  \item Если ваша реализация по сути является прохождением стандартного туториала,
  например, по отделению картинок кружек от котиков с помощью машинного обучения, то необходимо срочно сообщить об этом куратору на мат-мехе,
  иначе Государственная Экзаменацион\-ная Комиссия \enquote{порвёт Вас как Тузик грелку}, поставит \enquote{единицу},
  а все остальные Ваши сокурсники получат оценку выше. (Это не шутка, а реальная история 2020 года.)
\end{enumerate}

\noindent Если Вам предстоит защищать учебную практику, а эти рекомендации видятся как более подходящие для защиты ВКР, то ... отмаза не засчиты\-вается, сразу учитесь делать нормально.

% !TeX spellcheck = ru_RU
% !TEX root = vkr.tex

\section*{Заключение}
\textbf{Обязательно для промежуточного, полугодового, годового и  любых других отчётов.} Кратко, что было сделано.

\textbf{Для практик/ВКР.} Также важно сделать список результатов, который будет один к одному соответствовать задачам из раздела~\ref{sec:task}.

\begin{itemize}
\item Результат к задаче №1.
\item Результат к задаче №2.
\item и т.д.
\end{itemize}
\noindent Для промежуточных отчетов сюда важно записать какие задачи уже были сделаны за осенний семестр, а какие только планируется сделать.

Также сюда можно написать планы развития работы в будущем, или, если их много, выделить под это отдельную предпоследнюю главу.



\setmonofont{CMU Typewriter Text}
\bibliographystyle{ugost2008ls}
\bibliography{vkr}
\end{document}
