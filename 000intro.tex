% !TeX spellcheck = ru_RU
% !TEX root = vkr.tex

Больше информации можно поискать в слайдах Д.~Кознова~\cite{koznov}.

Тут  4 части (абзаца) максимум на 2 страницы:
\begin{enumerate}
\item Background, known information.
\item Knowledge gap, unknown information.
\item  Hypothesis, question, purpose statement.
\item Approach, plan of attack, proposed solution.
\begin{itemize}
\item Последний абзац должен читаться и быть понятным в отрыве от остальных трёх.
\end{itemize}
\end{enumerate}




\blfootnote{
	Дата сборки: \today\\
	Иногда рецензенту полезно знать какого числа компилировался текст, чтобы оценить актуальность версии текста. В этом случае полезно вставлять в текст дату сборки. Для совсем официальных релизов документа это не вполне канон.\\
Также здесь имеет смысл указать, если работа сделана на деньги, например, Российского Фонда Фундаментальных Исследований (РФФИ) по гранту номер такой-то, и т.п.}
