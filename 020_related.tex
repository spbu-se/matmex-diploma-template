% !TeX spellcheck = ru_RU
% !TEX root = vkr.tex

\section{Обзор (обязателен к новому году)}
\label{sec:relatedworks}
\emph{Обзор должен быть.} Здесь нужно писать, что индустрия и наука уже сделали по вашей теме. Он нужен, чтобы Вы случайно не изобрели какой-нибудь велосипед.

По-английски называется related works или previous works.

Если Ваша работа является развитием предыдущей и плохо понима\-ема без неё, то обзор должен идти в начале. Если же Вы решаете некоторую задачу новым интересным способом, то если поставить обзор в начале, то читатель может устать, пока доберется до вашего решения. В этом случае уместней поставить обзор после описания Вашего подхода к проблеме.

В обзоре необходимо ссылаться на работы других людей. В данном шаблоне задумано, что литература будет указываться в файле \verb=vkr.bib=. В нём указываются пункты литературы в формате \BibTeX{}, а затем на них можно ссылаться с помощью \verb=\cite{...}=. Та литература, на которую Вы сошлетесь, попадет в список литературы в конце документа. Если не сошлетесь~---  не попадёт. Спецификацию в формате \BibTeX{} почти никогда (для второго курса~--- никогда), не нужно придумывать руками. Правильно: находить в интернете описание цитируемой статьи\footnote{Например, \url{https://dl.acm.org/doi/10.1145/3408995} (дата доступа:   \DTMdate{2022-12-17}).},
копировать цитату с помощью кнопки \foreignquote{english}{Export Citation} и вставлять в \BibTeX{} файл. Если так не делать, но оформление литературы будет обрастать багами.
Например, \BibTeX{} по особенному обрабатывает точ\-ки, запятые и \verb=and= в списке авторов, что позволяет ему самому понимать, сколько авторов у статьи, и что там фамилия, что~--- имя, а что~--- отчество.

В обзоре и в остальном тексте вы наверняка будете использовать названия продуктов или языков программирования. Для них рекоменду\-ется (в файле \verb=preamble2.tex=) за\-дать специальные команды, чтобы писать сложные названия правильно и одинаково по всему доку\-менту. Написать с ошибкой  название любимого языка программирова\-ния науч\-ного руко\-водителя~--- идеальный вариант его выбесить.
